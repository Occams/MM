% ----------------------------------------------------------------------------
% ------------------------Bachelor Thesis
% ----------------------------------------------------------------------------
\documentclass[11pt]{beamer}
\usepackage[ngerman,english]{babel}
\usepackage[T1]{fontenc}
\usepackage[latin1]{inputenc}
\usepackage[vlined,ruled]{algorithm2e}
\usepackage{libertine,microtype,lipsum,csquotes,pgfplots,
        enumerate,amssymb,fixltx2e,listings,lastpage,fancybox,pgfplots}
\usepackage[scaled=.9]{inconsolata}
%\usepackage[nocut]{thmbox}
\usepackage{graphicx}

% -----------------------------------------------------
% ---------------- Package setup
% -----------------------------------------------------
%\hypersetup{pdfstartview={FitH}}

\lstset{
                breakautoindent=true,
                breakindent=2em,
                breaklines=true,
                tabsize=4,
                frame=blrt,
                frameround=tttt,
                captionpos=b,
                basicstyle=\scriptsize\ttfamily,
                keywordstyle={\color{SkyBlue}},
                %commentstyle={\color{OliveGreen}},
                stringstyle={\color{ScarletRed}},
                showspaces=false,
                %numbers=right,
                %numberstyle=\scriptsize,
                %stepnumber=1, 
                %numbersep=5pt,
                %showtabs=false
                prebreak = \raisebox{0ex}[0ex][0ex]{\ensuremath{\hookleftarrow}},
                aboveskip={1.5\baselineskip},
                columns=fixed,
                upquote=true,
                extendedchars=true
                }

% ------------------------------------------
% -------- xcolor - (Tango)
% ------------------------------------------

\definecolor{LightButter}{rgb}{0.98,0.91,0.31}
\definecolor{LightOrange}{rgb}{0.98,0.68,0.24}
\definecolor{LightChocolate}{rgb}{0.91,0.72,0.43}
\definecolor{LightChameleon}{rgb}{0.54,0.88,0.20}
\definecolor{LightSkyBlue}{rgb}{0.45,0.62,0.81}
\definecolor{LightPlum}{rgb}{0.68,0.50,0.66}
\definecolor{LightScarletRed}{rgb}{0.93,0.16,0.16}
\definecolor{Butter}{rgb}{0.93,0.86,0.25}
\definecolor{Orange}{rgb}{0.96,0.47,0.00}
\definecolor{Chocolate}{rgb}{0.75,0.49,0.07}
\definecolor{Chameleon}{rgb}{0.45,0.82,0.09}
\definecolor{SkyBlue}{rgb}{0.20,0.39,0.64}
\definecolor{Plum}{rgb}{0.46,0.31,0.48}
\definecolor{ScarletRed}{rgb}{0.80,0.00,0.00}
\definecolor{DarkButter}{rgb}{0.77,0.62,0.00}
\definecolor{DarkOrange}{rgb}{0.80,0.36,0.00}
\definecolor{DarkChocolate}{rgb}{0.56,0.35,0.01}
\definecolor{DarkChameleon}{rgb}{0.30,0.60,0.02}
\definecolor{DarkSkyBlue}{rgb}{0.12,0.29,0.53}
\definecolor{DarkPlum}{rgb}{0.36,0.21,0.40}
\definecolor{DarkScarletRed}{rgb}{0.64,0.00,0.00}
\definecolor{Aluminium1}{rgb}{0.93,0.93,0.92}
\definecolor{Aluminium2}{rgb}{0.82,0.84,0.81}
\definecolor{Aluminium3}{rgb}{0.73,0.74,0.71}
\definecolor{Aluminium4}{rgb}{0.53,0.54,0.52}
\definecolor{Aluminium5}{rgb}{0.33,0.34,0.32}
\definecolor{Aluminium6}{rgb}{0.18,0.20,0.21}
\definecolor{Brown}{cmyk}{0,0.81,1,0.60}
\definecolor{OliveGreen}{cmyk}{0.64,0,0.95,0.40}
\definecolor{CadetBlue}{cmyk}{0.62,0.57,0.23,0}

% ------------------------------------------
% -------- misc
% ------------------------------------------
\renewcommand*\oldstylenums[1]{{\fontfamily{fxlj}\selectfont #1}}

\usetheme{CambridgeUS}
\useinnertheme{rectangles}
\usecolortheme{beaver}
\AtBeginSection[]{\begin{frame}\frametitle{Table of Contents}\tableofcontents[currentsection]\end{frame}}
\beamertemplatenavigationsymbolsempty

% ------------------------------------------
% -------- hyphenation rules
% ------------------------------------------
\hyphenation{}

\begin{document}
%\newtheorem[L]{definition}{Definition}
\graphicspath{{img/}}

% Title stuff
\title[Copy-Move Detection]{Projekt 2}
\subtitle{Copy-Move Detection: Robust Match}
\author[Rainer, Huber, Watzinger]{Rainer Sebastian, Huber Bastian, Watzinger Daniel}
\institute[University of Passau]{Department of Informatics and Mathematics --- University of Passau}
\date{\today}
\subject{Informatics}

\frame{
        \titlepage
}

\section{Robust Match Algorithmen}
\begin{frame}
\frametitle{Robust Match Algorithmen}
	\begin{itemize}
		\item Es wurden 3 verschiedene Varianten des Robust Match Algorithmus implementiert.
		\item Die Berechnung der DCT-Koeffizienten f�r jeden Block und das anschlie�ende Sortieren tragen den gr��ten Teil zur Laufzeit bei.
		\item Bei allen Ans�tzen wurde die DCT-Berechnung parallelisiert.
    \end{itemize}
\end{frame}

\subsection{Naiv}
\begin{frame}
\frametitle{Robust Match Algorithmen -- Naiv}
	\begin{itemize}
		\item Berechnung der $16x16$ DCT-Matrizen mithilfe der vorgegebenen Formel.
		\item Vorberechnung aller konstanten Faktoren.
		\item Es werden genau $256 * 2 + 1$ FLOPS f�r ein Element der DCT-Koeffizientenmatrix ben�tigt.
		\item Die zu bearbeitenden Bl�cke werden unter den Threads aufgeteilt.
    \end{itemize}
\end{frame}

\subsection{Matrixmultiplikation}
\begin{frame}
\frametitle{Robust Match Algorithmen -- Matrixmultiplikation}
	\begin{itemize}
		\item Die $16x16$ DCT-Koeffizientenmatrizen werden durch Multiplikation einer speziellen Matrix $A$ von links und rechts gewonnen.
		\item	\begin{equation*}
					A * M * A = DCT
				\end{equation*}
		\item Insgesamt werden nur $64*2 + 1$ FLOPS f�r ein Element der DCT-Koeffizientenmatrix ben�tigt.
		\item Die zu bearbeitenden Bl�cke werden unter den Threads aufgeteilt.
    \end{itemize}
\end{frame}

\subsection{Partielle Berechnung}
\begin{frame}
\frametitle{Robust Match Algorithmen -- Partielle Berechnung}
	\begin{itemize}
		\item Idee: Die \emph{relevanten} Elemente der DCT-Koeffizientenmatrix stehen "links oben". Hier werden sich also die Bl�cke am wahrscheinlichsten unterscheiden.
		\item Berechnung der ersten $16$ Werte der DCT-Matrix jedes Blockes (Eine $4x4$ Matrix "links oben").
		\item Sortierung der Bl�cke und anschlie�endes Aussortieren von Bl�cken die keine Duplikate besitzen.
		\item Ermittlung der restlichen Elemente der Koeffizientenmatrizen f�r die Menge von Bl�cken die mindestens einen Partner besitzen.
		\item Diese ist je nach Bilddatei ca. halb so gro� wie die urspr�ngliche Menge an Bl�cken.
		\item Die zu bearbeitenden Bl�cke werden unter den Threads aufgeteilt.
    \end{itemize}
\end{frame}

\section{Ergebnisse}
\begin{frame}
\frametitle{Ergebnisse}
	\begin{itemize}
		\item Naiv: Sehr langsam, gute Erkennungsrate
		\item Matrixmultiplikation: Schnell, schlechte Erkennungsrate
		\item Partielle Berechnung: Sehr schnell, selbe Erkennungsrate wie der naive Algorithmus
	\end{itemize}
\end{frame}

\section{Graphical User Interface}
\begin{frame}
\frametitle{Graphical User Interface}
	Funktionalit�t:
	\begin{itemize}
		\item Bilddatei laden/�ffnen, Auswahl einer Implementierung des Algorithmus, Starten/Abbruch des ausgew�hlten Algorithmus
		\item Multithreading (de)aktivieren, Anpassen der Parameter \texttt{quality} und \texttt{threshold} 
		\item Nachdem eine Bilddatei bearbeitet wurde kann zwischen einer Darstellung des Originalbilds, der Hervorhebung der ermittelten Regionen und einer Visualisierung der Shiftvektoren gewechselt werden.
		\item Die ermittelten Shiftvektoren k�nnen anhand ihrer L�nge dynamisch gefiltert werden.
    \end{itemize}
\end{frame}

\section*{Fragen}
\begin{frame}
        \frametitle{Fragen}
        \begin{center}
                \includegraphics[keepaspectratio=true,width=0.6\textwidth]{question-mark}
        \end{center}
\end{frame}

\end{document}